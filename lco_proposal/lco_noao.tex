% NOAOPROP.TEX -- template for NOAO STANDARD telescope proposals.
% Revised for the 2018B semester (August - January 2018)
% For later semesters, the current form may be obtained from our Web
% page at http://www.noao.edu/noaoprop/noaoprop.html

% DEADLINE FOR SUBMISSIONS: 11:59pm MST MONDAY APRIL 2 

% This form can NOT be used to submit Gemini proposals.  Please
% see:   http://www.noao.edu/noaoprop/help/pit.html

% This LaTeX template has been returned to you upon request through
% our Web-based proposal form.  The template has been customized for
% you based on the run information that you entered via the Web form.
% As an option you may complete this form locally and submit it by
% email following the instructions below.  If the run information is
% incomplete you should go back to the Web form, complete all the
% suggested run information and then obtain another customized 
% template before proceeding. 
%
% WEB FORM SUBMISSION:
% If you have not modified this document locally, we encourage web
% submissions using the "Proposal Submission" button on the web
% proposal form.  Be sure to click on the second prompt on the
% following page to complete submission.   Skip to step #3 below
% for some additional information.
%
% FILE UPLOAD SUBMISSION:
% If you have modified this document locally and are ready to submit, 
% you do not need to return to the web form.  Instead, you may submit
% this template and figure files by following the instructions 
% below.
%
%   1. Before submitting this form electronically, run it through latex
%      and print it out to make certain that it looks the way that you
%      wish the review panel to see it.
%   
%   2. Upload THIS file (NOT the PostScript or PDF version) and
%      all figures on the web at http://www.noao.edu/noaoprop/submit/
%
%   3. If the proposal is a thesis or if the principal investigator is a
%      a graduate student, the student's faculty advisor must complete
%      the online form at http://www.noao.edu/noaoprop/thesis/
%      This form should be completed within two business days after the
%      deadline.  Graduate students proposing thesis observations should
%      consult NOAO policies concerning thesis programs and travel support.
%
%   4. QUESTIONS?  If you have questions about submitting your proposal
%      send email to noaoprop-help@noao.edu.  Information about
%      instrumentation can be found at the NOAO Web 
%      site at http://www.noao.edu/noaoprop/noaoprop.html.  Specific 
%      instrumentation or facility questions for CTIO or KPNO
%      can also be sent to the respective sites at ctio@noao.edu, 
%      and kpno@noao.edu
%      Gemini questions can be  emailed to the US mirror scientists 
%      at nssc@noao.edu.
%
%   5. When your proposal is received at NOAO you will be sent an
%      automatic email message verifying its receipt along with a
%      proposal ID number.  If you do not receive this message within
%      15 minutes of the time you sent your proposal send email to
%      noaoprop-help@noao.edu for assistance.  You may track your
%      proposal processing by the proposal ID number at the following
%      web page:   https://www.noao.edu/cgi-bin/noaoprop/propstatus
% ___________________________________________________________________
% THE FORM STARTS HERE
%

% Please do not modify or delete this line.
\documentclass[11pt]{article}
\usepackage{nprop30}


% Please do not modify or delete this line.
\begin{document}

% Uncomment the following line and enter a previous semester and ID
% (e.g. 01B-0987) if you wish to flag this proposal as a resubmission
%\pastid{}

% Please do not modify or delete this line.
\proposaltype{Standard}
% Do not delete the following uncommented line which is required for Gemini 
% proposals.  This line indicates whether or not the proposal is being 
% submitted to multiple partner countries - only "yes" or "no" is valid.
\geminidata{mpartners}{no}

%%%%%%%%%%%%%%%%%%%%%%%%%%%%%%%%%%%%%%%%%%%%%%%%%%%%%%%%%%%%%%%%%%%%%

% SCIENTIFIC CATEGORIES
%
% Please select a "scientific category" that best describes your
% program by uncommenting only ONE of the selections below.  Your
% \sciencecategory selection will be used to assign a review panel to
% your proposal.  DO NOT MODIFY THE SELECTION YOU UNCOMMENT.  A
% description of each of these categories is available on our Web page
% at https://www.noao.edu/noaoprop/help/scicat.html

% EXTRA-GALACTIC LIST (do not uncomment this line)
%\sciencecategory{Active Galaxies}
%\sciencecategory{Cosmology}
%\sciencecategory{Large Scale Struc.}
%\sciencecategory{Clusters of Galaxies}
%\sciencecategory{High Z Galaxies}
%\sciencecategory{Low Z Galaxies}
%\sciencecategory{Resolved Galaxies}
%\sciencecategory{Stellar Pops (EGAL)}
%\sciencecategory{EGAL - Other}

% GALACTIC/LOCAL GROUP LIST (do not uncomment this line)
%\sciencecategory{Star Clusters}
%\sciencecategory{Stellar Pops (GAL)}
%\sciencecategory{HII Reg., PN, etc.}
%\sciencecategory{ISM}
%\sciencecategory{Star Forming Regions}
%\sciencecategory{Young Stellar Obj.}
%\sciencecategory{Massive Stars}
%\sciencecategory{Low Mass Stars}
%\sciencecategory{Stellar Remnants}
%\sciencecategory{Galactic - Other}

% SOLAR SYSTEM LIST (do not uncomment this line)
%\sciencecategory{Kuiper Belt Objects}
\sciencecategory{Small Bodies \& Moons}
%\sciencecategory{Planets}
%\sciencecategory{Extrasolar Planets}
%\sciencecategory{Solar System - Other}

%%%%%%%%%%%%%%%%%%%%%%%%%%%%%%%%%%%%%%%%%%%%%%%%%%%%%%%%%%%%%%%%%%%%%%

% TITLE
%
% Give a descriptive title for the proposal in the \title command.
%
% Note that a title can be quite long; LaTeX will break the title into
% separate lines automatically.  If you wish to indicate line breaks
% yourself, do so with a `\\' command at the appropriate point in
% the title text.  Use both upper and lower case letters (NOT ALL CAPS).

\title{Lightcurves of Main Belt Asteroids}


%%%%%%%%%%%%%%%%%%%%%%%%%%%%%%%%%%%%%%%%%%%%%%%%%%%%%%%%%%%%%%%%%%%%%%

% ABSTRACT
%
% Give a general abstract of the scientific justification appropriate
% for a non-specialist.  Write between the \begin{abstract} and 
% \end{abstract} lines.  Limit yourself to approximately 175 words.
% Abstracts of accepted proposals will be made publicly available.

% DO NOT remove the \begin{abstract} and \end{abstract} lines.

\begin{abstract}
We propose to obtain dense, continuous time sampling of 10 main belt asteroids with LCO as follow up to sparse light curve sampling obtained with ZTF. The extremely well-characterized light curve segment from LCO will serve to validate Gaussian Process techniques we are developing to analyze sparse asteroid light curves from surveys like ZTF or LSST.

These MCMC + Gaussian Process techniques provide formal error estimates on asteroid rotation periods, as well as allowing us to account for non-sinusoidal period profiles, and (in future work) fit for colors, phase curves and shape models when using survey data from a multi-band, long-term survey such as ZTF or LSST. Obtaining LCO data to supplement the ZTF data while developing these algorithms will provide invaluable real-world tests of our methods.

\end{abstract}

%%%%%%%%%%%%%%%%%%%%%%%%%%%%%%%%%%%%%%%%%%%%%%%%%%%%%%%%%%%%%%%%%%%%%%

% SUMMARY OF OBSERVING RUNS REQUESTED
%
% List a summary of the details of the observing runs being requested,
% for UP TO SIX runs.  The parameters for each run are segregated
% between \begin{obsrun} and \end{obsrun} lines.  Please be sure
% that the information is isolated properly for each run.
%
%   \begin{obsrun}
%   \telescope{}        % For example, \telescope{KP-4m}
%   \instrument{}       % For example, \instrument{ECHUV + T2KB}
%   \numnights{}        % For example, \numnights{6}
%   \lunardays{}        % For example, \lunardays{grey}
%   \optimaldates{}     % For example, \optimaldates{Sep - Nov}
%   \acceptabledates{}  % For example, \acceptabledates{Aug - Jan}
%   \end{obsrun}
%
% The following telescope identifiers MUST be used in the \telescope{}
% field.  Some of the telescope identifiers must include an observatory
% code as well.
%
% CTIO: CT-4m, SOAR, CT-1.3m, CT-0.9m
% KPNO: WIYN, KP-0.9m
% AAO: AAT
% LCO: LCO-2m, LCO-1m
% CHARA: CHARA
%
% Select the instrument and detector identifiers from the list on our
% Web page at https://www.noao.edu/noaoprop/help/facilities.html.
% The correct codes MUST be used to ensure your correct
% instrument + detector combination.
%
% \numnights should give the number of nights of the run (for queue
% observations, use fractional 10-hour equivalent nights, e.g.
% 3 hours = 0.3 nights).  Formats such as 5x0.5 are acceptable.
%
% \lunardays should contain the word "darkest", "dark", "grey", or
% "bright", which in turn reflects the number of nights from new moon
% where darkest<=3, dark<=7, grey<=10, bright<=14.  Particular lunar 
% phase requirements dictated by the science program (e.g., "<=12", 
% "+9, -6", or "full moon more than 2 hours away from Taurus") should
% be noted in the "scheduling constraints or non-usable dates" section 
% below.
%
% \optimaldates should contain the range of OPTIMAL months, as shown 
% below.
%
% \acceptabledates should give the range of ACCEPTABLE months (i.e.,
% you would not accept time outside those limits).
% NOTE THAT DUE TO INSTRUMENT BLOCKING RESTRICTIONS YOU SHOULD MAKE 
% THIS RANGE AS GENEROUS AS POSSIBLE.
%
% For QUEUE-SCHEDULED observations, you may set the date range to the 
% full semester range and set \lunardays to the brightest moon your
% observations could tolerate if the program were scheduled classically.
%
% To enter the acceptable and optimal date ranges, please use two
% dash-separated months with 3-letter abbreviations for the month
% (Jan, Feb, Mar, Apr, May, Jun, Jul, Aug, Sep, Oct, Nov, Dec).
% For example:  \optimaldates{Nov - Dec}.
% We appreciate your help in not using vague range specifications
% like "October dark run" or "mid-January" which will require human
% intervention.
%
% FOR LONGTERM STATUS PROPOSALS SPECIFY ONLY THE RUNS FOR THE CURRENT
% SEMESTER, AND NOT FOR ANY SUBSEQUENT SEMESTERS.

% DO NOT remove any of the \begin{obsrun} and \end{obsrun} blocks, 
% even if the blocks are empty.

\begin{obsrun}
\telescope{LCO-1m}
\instrument{Sinistro}
\numnights{101 hrs}
\lunardays{dark}
\optimaldates{Jun - Nov}
\acceptabledates{Jun - Nov}
\end{obsrun}

\begin{obsrun}
\telescope{}
\instrument{}
\numnights{}
\lunardays{}
\optimaldates{}
\acceptabledates{}
\end{obsrun}

\begin{obsrun}
\telescope{}
\instrument{}
\numnights{}
\lunardays{}
\optimaldates{}
\acceptabledates{}
\end{obsrun}

\begin{obsrun}
\telescope{}
\instrument{}
\numnights{}
\lunardays{}
\optimaldates{}
\acceptabledates{}
\end{obsrun}

\begin{obsrun}
\telescope{}
\instrument{}
\numnights{}
\lunardays{}
\optimaldates{}
\acceptabledates{}
\end{obsrun}

\begin{obsrun}
\telescope{}
\instrument{}
\numnights{}
\lunardays{}
\optimaldates{}
\acceptabledates{}
\end{obsrun}


% If there are scheduling constraints or non-usable dates for any of
% the runs specified, (i.e., other than the default lunar phase 
% requirements or when your object is up) please give the dates by 
% filling in the curly braces in \unusabledates{}.  Note here if you 
% are requesting runs in an "either/or" situation, e.g. run 1 or run 2, 
% but not both. This is also the place to advise us of any special 
% constraints which affect the scheduling of your observing run (e.g. 
% "schedule run #1 before run #2" or "run dates must be coordinated 
% with HST observations").
% 
% Please limit your text to six lines on the printed copy.

\unusabledates{}

%%%%%%%%%%%%%%%%%%%%%%%%%%%%%%%%%%%%%%%%%%%%%%%%%%%%%%%%%%%%%%%%%%%%%%%

% INVESTIGATOR'S (PI AND CoI) INFORMATION BLOCKS
%
% Please give the PI's name (first name first followed by middle
% initial and last name), affiliation, department and complete mailing
% address, as well as an email address.  Also give a complete phone
% number, and a number for a fax machine if you have access to one.
% You must also indicate the principal investigator's status with one
% of the one-letter codes inside the \invstatus{} curly braces, as
% indicated below.
%
% The affil{}, \department{}, \address{} (use a comma separate list as
% needed), \city{}, \state{}, \zipcode{}, and \country{} (for non-US
% addresses) fields will be used together as your full postal mailing
% address.  Please be sure this information is complete.  Note that
% some institutions will not deliver postal mail if a department is
% not included in the postal mailing address.  Non-US addresses should
% include the country and any local postal codes.
%
% The fax number does not print on the form.
%
% For each CoI please include a name, affiliation, email address and
% investigator's status within the \begin{CoI} and \end{CoI} lines.
%
% For each \invstatus{} field, please fill in the appropriate
% investigator status code from the following list.  If the investigator
% is a graduate student, indicate "T" if THIS proposal is related to a
% thesis project, or "G" otherwise.  This code should represent the
% status of the individuals at the time of the proposal submission.
% This information is necessary to assist us with our required reporting
% to the NSF.
%
% \invstatus{P} % investigator has obtained PhD or equivalent
% \invstatus{T} % investigator is grad student, proposal is thesis
% \invstatus{G} % investigator is grad student, proposal not thesis
% \invstatus{U} % investigator is an undergraduate student
% \invstatus{O} % investigator has other status (none of the above)
%
% DO NOT remove the \begin{PI} and \end{PI}.  Only one individual's
% name per \name field is allowed.
%
% Investigator names will now appear on page 2 of the printed proposal.
% Do not remove this line.
\investigators

\begin{PI}
\name{Lynne Jones}
\affil{University of Washington}
\department{Department of Astronomy}
\address{Box 351580, U.W.}
\city{Seattle}
\state{WA}
\zipcode{98105}
\country{USA}
\email{lynnej@uw.edu}
\phone{2067954755}
\fax{}
\invstatus{P}
\end{PI}

\begin{CoI}
\name{D. Huppenkothen}
\affil{University of Washington}
\email{d.huppenkothen@gmail.com}
\invstatus{P}
\end{CoI}

\begin{CoI}
\name{S. Greenstreet}
\affil{B612 Asteroid Institute / DIRAC Institute}
\email{sarah@b612foundation.org}
\invstatus{P}
\end{CoI}

\begin{CoI}
\name{B. Bolin}
\affil{B612 Asteroid Institute / DIRAC Institute}
\email{bolin@b612foundation.org}
\invstatus{P}
\end{CoI}

\begin{CoI}
\name{M. Juric}
\affil{University of Washington}
\email{mjuric@astro.washington.edu}
\invstatus{P}
\end{CoI}


%%%%%%%%%%%%%%%%%%%%%%%%%%%%%%%%%%%%%%%%%%%%%%%%%%%%%%%%%%%%%%%%%%%%%%

% In the following "essay question" sections, the delimiting pieces of
% markup (\justification, \expdesign, etc.) act as LaTeX \section*{}
% commands.  If the author wanted to have numbered subsections within
% any of these, LaTeX's \subsection could be used.
%
% DO NOT REDUCE THE FONT SIZE, and do not otherwise fiddle with the
% format to get more on a page.  We will reset any changes back to the
% default font.

% SCIENTIFIC JUSTIFICATION
%
% Give the scientific justification for the proposed observations.
% This section should consist of paragraphs of text followed by any
% references and up to three figures and captions.  Be sure to include
% overall significance to astronomy.  THE SCIENTIFIC JUSTIFICATION
% SHOULD BE LIMITED TO ONE PAGE (the review panels have requested that
% we not send them more than one page), with up to two additional pages
% for references, figures (no more than three), and captions. 

% If you wish to use our "reference" environment, follow the following
% example (journal commands are compatible with AASTeX v4.0):
%
%\begin{references}
%\reference Armandroff \& Massey 1991 \aj, 102, 927.
%\reference Berkhuijsen \& Humphreys 1989 \aap, 214, 68.
%\reference Massey 1993 in Massive Stars: Their Lives in the 
% Interstellar Medium (Review), ed. J. P. Cassinelli and E. B. 
% Churchwell, p. 168.
%\reference Massey \& Armandroff 1999, in prep.
%\end{references}

% Only EPS figures may be included with your proposal.  In order to
% include an EPS plot, you should use the LaTeX "figure" environment.
% The plot file is included with the \plotone{FILENAME} command; two
% side-by-side plot files can be included by typing
% \plottwo{FILENAME1}{FILENAME2}.  Use \caption{} to specify a caption.
% The \epsscale{} command can be used to scale \plotone plots if they
% appear too large on the printed page.  Contact us if you have any
% figure questions or encounter any problems with figures
% (noaoprop-help@noao.edu).
%
% \begin{figure}
% \epsscale{0.85}
% \plotone{sample.eps}
% \caption{Sample figure showing important results.}
% \end{figure}
%
% If you need to rotate or make other transformations to a figure, you
% may use the \plotfiddle command:
% \plotfiddle{PSFILE}{VSIZE}{ROTANG}{HSCALE}{VSCALE}{HTRANS}{VTRANS}
% \plotfiddle{sample.eps}{2.6in}{-90.}{32.}{32.}{-250}{225}
% where HSCALE and VSCALE are percentages and HTRANS and VTRANS are
% in PostScript units, 72 PS units = 1 inch.
%
% Note that the Web form provides several useful and simple figure 
% options.

\sciencejustification
With the start of all-sky synoptic surveys (PanSTARRS, ZTF, LSST), new opportunities
are opening up for studying some of the smallest members of our Solar System: 
Near Earth and Main Belt Asteroids. These asteroids provide keys to understanding
the formation and evolution of our Solar System, entangled in their distributions of
sizes, shapes, spins and physical structure. We can learn about these properties from
the details of the light curve information available in the survey data, but it requires processing
tens of thousands of sparsely sampled light curves to fit the asteroid rotation periods, 
the profiles of the asteroid light curves, phase curve variations and the asteroid colors,
including variations with phase angles and rotation.

With any individual asteroid, the amplitude and period of the light curve can be used
to derive the rotation period and estimate the elongation. Together, these can be used
to constrain the structural properties of the asteroid (rubble pile vs. monolithic rock). 
The light curve profile can be used to determine the 
asteroid's shape, particularly when data is obtained over a wide variety of viewing geometries
and phase angles. Observations over a wide range of phase angles must also fit for the phase curve
of the asteroid, constraining the surface properties (smooth vs rough). The rotation period and shape can 
be used to determine the full spin state of the asteroid. The distribution of all of these parameters 
constrains models of the collisional evolution of the asteroids. Asteroid colors (and any variations with 
phase angle or rotation) must also be accounted for when using the non-simultaneous multi-band 
observations typical of all-sky synoptic surveys. The colors constrain the chemical composition of the 
asteroid, the distribution across the population then constrains models of asteroid formation.

Current methods of fitting asteroid sparse light curves fall under the categories of
fourier analysis of the light curve (such as with Lomb-Scargle or FALC) or non-parametric
minimization algoriths (such as Phase Dispersion Minimization (PDM) or SuperSmoother)  
(see {\it e.g.} Waszczak et al, \aj\, 2015, Polishook et al,
MNRAS 2012, Harris et al, Icarus 2012, Warner \& Harris, Icarus 2011,
Masiero et al, Icarus 2009, Mann et al, \aj\, 2007).  With fourier analysis, the 
light curve is approximated as a series (for sparse light curves, typically two harmonics)
of sine curves and the best fit period and amplitude is determined using a chi-squared minimization. 
PDM is the non-parametric method most typically used for asteroids; here the measured magnitudes
are folded at a range of phases and the scatter within each phase bin is measured to search for 
a minimum. A drawback to both of these methods, however, is that the fitting process makes the true 
uncertainty of the fit parameters (the period, amplitude and shape of the light curve profile) extremely hard
to calibrate. This is problematic when attempting to draw conclusions about the population-level distribution
of parameters.

We are developing techniques to fit sparse light curves using MCMC and Gaussian Process
methods. The advantage to these techniques is that they provide a generative model;
the formal uncertainty in the fit parameters can be well determined. Further advantages
include the capability of including fits for the phase function, potential tumbling, 
and multi-color observations. These techniques will be necessary to fully
exploit the large amounts of sparse light curve data coming from surveys like
PanSTARRS, ZTF and LSST. We will develop and test our process with simulated light curves and against 
archive data. However, simulated data may not sample the full range
of true light curve profiles and archive data will generally not be obtained
during the same apparition as the sparsely sampled data obtained with ZTF, thus may
not present the same light curve profile. Obtaining real, densely sampled
light curve data will provide an invaluable resource to fully test our methods.


\clearpage

% \begin{figure}
% \epsscale{0.85}
% \plotone{sample.eps}
% \caption{Sample figure showing important results.}
% \end{figure}



% EXPERIMENTAL DESIGN
%
% This section should consist of text only (no figures).
% There is a limit of one page of printed text.

% Describe your overall observational program.  How will these 
% observations contribute toward the accomplishment of the goals 
% outlined in the science justification?  If you've requested 
% long-term status, justify why this is necessary for successful 
% completion of the science.
%
% NOTE: In previous versions of the proposal form, this section
% requested details about the use of non-NOAO observing facilities. 
% Such information should now be entered in the following "Other
% Facilities" section.

\expdesign
The observations we are proposing to obtain with LCO are intended to
densely sample the light curve, following up on the sparse sampling
obtained with ZTF.

We will use our Gaussian Process methods on the sparsely sampled data
obtained with ZTF, but need densely sampled light curves over a full
rotation period to verify the results, especially non-sinusoidal
features in the light curve profile. 

We expect ZTF to observe more than a thousand main belt asteroids
brighter than $r=20$ in the region between 290 and 0 degrees RA and 
+10 and -30 degrees Declination, consistent with the footprint of the
ZTF survey, avoiding the galaxy, and suitable for LCO observations with
the 1-m network between June and Dec 2018. This provides plenty of
potential targets, with a range of expected light curve periods,
profiles and amplitudes. We will choose a sample of approximately 10
objects which provide the best challenge to the Gaussian Process method
development, with a range of periods between 2 and 10 hours. 

To gather data to fully sample the light curve over the entire rotation
period, and to maintain sensitivity to non-sinusoidal elements in the
light curve, we need to continually observe the targets for at least
two light curve periods. By sampling two light curve periods, we can
remove ambiguities due to the possibility of either the light-curve
originating from a rotation period (a `double-peaked' light curve) or
albedo variations (a `single-peaked' light curve).  By continually
observing the object, we are sensitive to non-sinusoidal variations in
the light curve profile. LCO is unique in its capabilities to
continually observe a target for longer than a few hours, due to its
global distribution of telescopes, and is thus an ideal resource for
this project. Although we can recover from missed observations due to
local weather conditions, the ideal situation would be to observe each
target continually throughout a 4-20 hour time period (depending on the
rotation period of each asteroid). 

A SNR of about 20 is required in order to obtain photometry with
uncertainties small enough to accurately determine the light curve. At
r=20, with Sinistro used on the 1-m telescopes, this is possible to
achieve with exposure times of about 176 seconds during `half' bright
time (estimated using the LCO Exposure Time Calculator) or 124 seconds
during dark time, which provides a reasonable sampling rate (218
seconds, including readout) for these multi-hour light curves. 

Depending on the light curve profile and possible ambiguities between
light curve period and rotation period, we may request alternating
observations in g and r band. For targets with no ambiguity between
double- and single- peaked light curves, but with complexity in their
light curve profile, we will observe in just r band.


% PROPRIETARY PERIOD
% 
% Enter the proprietary period for your data between the braces.
% The normal duration is 18 months from when the data are taken at
% the telescope.  Requests for longer proprietary periods must
% be approved by the NOAO Director.

\proprietaryperiod{12 months}


% OTHER FACILITIES OR RESOURCES
% 
% This section should consist of text only (no figures).
% Please limit to about a half page of printed text.
% 
% 1) We are interested in understanding how observations made through
% NOAO observing opportunities complement or support data from other
% facilities both on the ground and in space.   We will use this
% information to guide the evolution of the NOAO program; it will not
% affect the success of your proposal in the evaluation process.
% Please describe how the proposed observations complement data from  
% other facilities, including private observatories and both ground-
% and space-based telescopes.  In addressing this question, take a
% broad view of your research program.  Are the data to be obtained 
% through this proposal going to help select samples for detailed
% observations using larger telescopes or from space observatories?
% Are these data going to be directly combined with data obtained
% elsewhere to test a hypothesis?  Will these observations have
% relevance to other observations, even though the proposal stands
% on its own?  For each of these other facilities, indicate the nature
% of the observations (yours or those of others), and describe the
% importance of the observations proposed here in the context of the
% entire program.
%
% 2) Do you currently have a grant that would provide resources
% to support the data processing, analysis, and publication of the
% observations proposed here?

\otherfacilities
We are proposing to obtain dense light curve coverage with LCO to
complement sparse, LSST-style coverage obtained with the Zwicky
Transient Factory, ZTF. The ZTF observations are expected to be
obtained approximately every three or four days, with two observations
in each night separated by approximately 30 minutes, continually over
several months. The University of Washington is a ZTF partner, thus we
can access both the public data (sampled at the rate described) and
the partnership data, which has the possibility of providing a
long-term light-curve sampled at a slightly higher rate (6 times per
night instead of 2). This will help us understand the level of
``sparse-ness'' which works best when combining dense and sparse
measurements.

% LONG-TERM DETAILS
%
% If you are requesting long-term status for this proposal briefly 
% state the requirements for telescope time (telescope, instrument, 
% number of nights) needed in subsequent semesters to complete this 
% project in \longtermdetails (be sure to uncomment the 
% \longtermdetails line below).
%
% If this is a long-term request you MAY ALSO NEED to modify the
% \proposaltype keyword at the top of this form changing "Standard"
% to "Longterm" (where is says "Please do not modify or delete this 
% line!").  It is this keyword that will flag this proposal as a 
% long-term status request, regardless of what may be entered here 
% in \longtermdetails!
%
% If this is not a long-term status request then please ignore this
% section.
%
%
%\longtermdetails


% PAST USE
%
% How effectively have you used the facilities available through NOAO
% in the past?
% List allocations of telescope time on facilities available through 
% NOAO to the Principal Investigator during the past 2 years, together 
% with the current status of the data (cite publications where 
% appropriate).  Mark any allocations of time related to the current 
% proposal with a \relatedwork{} command.

\thepast

%%%%%%%%%%%%%%%%%%%%%%%%%%%%%%%%%%%%%%%%%%%%%%%%%%%%%%%%%%%%%%%%%%%%%%

% OBSERVING RUN DETAILS - REQUIRED FOR EACH OBSERVING RUN REQUESTED
%
% For each run requested earlier in the \begin{obsruns}-\end{obsruns}
% sections of this proposal form, further run information must be
% specified.  Enter this block of information for each non-Gemini run.

% The \runid field must contain the run number plus 
% telescope/instrument-detector information as it appears for each 
% run in the obsruns sections. For example, 
% \runid{1}{KP-4m/ECHUV + T2KB}.

\runid{1}{LCO-1m/Sinistro}

% Describe the observations to be made during this observing run in
% the \technicaldescription section. Justify the specific telescope,
% the number of nights, the instrument, and the lunar phase requested.
% List objects, coordinates, and magnitudes (or surface brightness, if
% appropriate) using a LaTeX-coded table in this section or optionally
% enter the target information using the Target Tables described at
% the end of this template.  Target Tables of objects are required for 
% queue runs.
%
% Exposure time calculators (ETCs) for some optical and IR 
% instruments in use at CTIO and at KPNO are available to assist 
% you with the preparation of your proposal.  See the Web pages:
%    Imaging ETC      - http://www.noao.edu/gateway/ccdtime/
%    Spectroscopy ETC - http://www.noao.edu/gateway/spectime/

\technicaldescription
We have requested dark or moon-down gray time for this proposal, as our
targets will be in the ecliptic and thus near the moon.

For each of our targets, we request observations covering a
full rotation period (twice the light curve period). We will determine
periods and light curve profiles for potential targets using Gaussian
Process methods on the sparse ZTF data. We will select a set of
approximately 10 targets with periods between 2 and 10 hours which
provide challenges to the Gaussian Process technique for observation
with LCO. Choosing 10 targets gives the opportunity to select a reasonable range
of test cases. The chosen targets will be $r\le20$, with an average 
light curve period of 5 hours.

A SNR of 20 will allow us to resolve non-sinusoidal small amplitude
features in the light curve profile, and can be reached in a single
exposure on the LCO 1-m with Sinistro in 176 seconds during
`half' bright time or 124 seconds during dark time (estimated from the
LCO ETC). Including a readout + overhead time of 42 seconds per
exposure, we can sample the asteroid lightcurve on a timescale of 166
to 218 seconds, which provides a reasonable rate for these multi-hour
light curve periods.

For each asteroid, we would observe in a sequence of $\approx200$s
exposures, in either just $r$ band or in alternating $r$ and $g$,
depending on the details of the sparse light curve data obtained from
ZTF. Ideally the sequence would continue continuously throughout two light curve periods
(as fit from the ZTF data) to unambiguously sample a full rotation
period, but non-continuous coverage planned out over the entire rotation 
can be used if necessary. If filter changes are required, the overhead for the filter change will
be absorbed in the sampling rate, without changing the total 
amount of time required to cover the light curve. Observing a target for 
10 continuous hours (corresponding to a light curve period of 5 hours) with an airmass $<1.5$
would require observations acquired from about three or four locations, adding approximately
6 minutes of slew settle and acquisition time per target.

Across all 10 targets, with an average light curve period of 5 hours (requiring observations
spanning 10 hours) and an addition 6 minutes of overhead, we are asking for 101 hours of time.


% Several instrument configuration parameters are requested.  Fill
% these in as appropriate for each run. 
%
% \begin{configuration}
% \filters{}            % List here any filters that you plan to use.
% \grating{}            % List any gratings/grisms need with this run.
% \order{}              % Specify any grating order(s).
% \crossdisperser{}     % List cross disperser, if needed.
% \slit{}               % Enter slit widths you plan to use.
% \multislit{}          % yes or no only
% \wstart{}             % Starting wavelength of wavelength range.
% \wend{}               % Ending wavelength of wavelength range.
% \cable{}              % For CTIO/Hydra: enter 2.0".  For WIYN: enter
%                         red, blue, or densepak.
% \corrector{}          % Enter red or blue for KP-coude, CAM5.
% \collimator{}         % Enter collimator needed.
% \adc{}                % If user selectable, enter yes or no only.
% \end{configuration}
%
% Details about these fields are available in the online help for the
% Web form at http://www.noao.edu/noaoprop/help/standard.html#iconfig

\begin{configuration}
\filters{r, g}
\grating{}
\order{}
\crossdisperser{}       
\slit{}
\multislit{}            
\wstart{}
\wend{}
\cable{}
\corrector{}            
\collimator{}             
\adc{}
\end{configuration}



% COORDINATE RANGES OF PRINCIPLE TARGETS   
% Use the \targetsra and \targetsdec fields to specify the range
% of right ascension (in hours) and declination (in degrees) of your
% principle targets for this observing run.
%
% For example:
%
%  \targetsra{14 to 17}
%  \targetsdec{-10 to 35}

\targetsra{20 to 2}			% RA range in hours
\targetsdec{10 to -30}			% Dec. range in degrees



% Use \specialrequest to describe briefly any special or non-standard
% usage of instrumentation. 
\specialrequest


% If you plan to submit any Target Tables for this run they must be
% entered here.  Target Tables are required for all queue runs.

% TARGET TABLES
%
% Target Tables are required for all queue runs and are
% are optional for classically scheduled runs.  Target
% Tables, if included, are associated with each run in the observing
% run details section of the proposal form and must follow the 
% \specialrequest in each section.
%
% CT-1.3m tables require that all the fields in the target table be
% specified.  For other tables specific fields may be deleted EXCEPT 
% for the \obscomment command as mentioned below.
%
% Queue investigators should supply good coordinates
% with the \ra and \dec commands since these values may be used to
% observe your fields.
%
% Note that for iterative targets, only the parameters that need
% to be changed have to be specified.  Once a parameter is specified in
% a targettable environment, it is retained until explicitly changed.
%
% The \obscomment command is REQUIRED for each target entry and
% must be the last item; this command forces each target line to be
% printed.  If no comment is needed, leave the argument blank.
%
% The \begin{targettable}{} command for each table must contain the
% telescope/instrument-detector information for that particular run,
% i.e,. \begin{targettable}{KP-4m/ECHUV + T2KB}.
%
% HINTS: Long tables do not break across pages. If it is necessary to
% continue a table across a page you must start a new table.  Use
% /clearpage before the \begin{targettable} command for the new table.
%
%\begin{targettable}{}
%\objid{}           % specify a 3-digit number for each target.
%\object{}          % 20 characters maximum
%\ra{}              % e.g., xx:xx:xx.x
%\dec{}             % e.g., +-xx:xx:xx.x
%\epoch{}           % e.g., 1950.3
%\magnitude{}
%\filter{}
%\exptime{}         % in seconds PER EXPOSURE
%\nexposures{}      % Number of exposures
%\moondays{}        % Days from new moon, use a number 0-14
%\skycond{}         % "spec" or "phot"
%\seeing{}          % max allowable PSF FWHM (arcsecs)
%\obscomment{}      % 20 characters maximum - REQUIRED COMMAND
%  - repeat target entry parameters as needed to complete Table -
%\end{targettable}


%%%%%%%%%%%%%%%%%%%%%%%%%%%%%%%%%%%%%%%%%%%%%%%%%%%%%%%%%%%%%%%%%%

% Detailed information as noted below must be provided for all runs 
% up to SIX runs.  Since Gemini runs must be specified through the 
% Web form this section only applies to non-Gemini runs - use the 
% Web form to generate detailed information and target tables for 
% Gemini runs.  Target Tables are required for queue runs but
% are optional for classical observing.
%
%\runid{}{}
%\technicaldescription
%\begin{configuration}
%\filters{}
%\grating{}
%\order{}
%\crossdisperser{}
%\slit{}
%\multislit{}
%\wstart{}
%\wend{}
%\cable{}
%\corrector{}
%\collimator{}
%\adc{}
%\end{configuration}
%\specialrequest    % remove and not use for HET runs
%\begin{targettable}{}
%\objid{}           % specify a 3-digit number for each target
%\object{}          % 20 characters maximum
%\ra{}              % e.g., xx:xx:xx.x
%\dec{}             % e.g., +-xx:xx:xx.x
%\epoch{}           % e.g., 1950.3
%\magnitude{}
%\filter{}
%\exptime{}         % in seconds PER EXPOSURE
%\nexposures{}      % Number of exposures
%\moondays{}        % Days from new moon, use a number 0-14
%\skycond{}         % "spec" or "phot"
%\seeing{}          % max allowable PSF FWHM (arcsecs)
%\obscomment{}      % 20 characters maximum - REQUIRED COMMAND
%  - repeat target entry parameters as needed to complete Table -
%\end{targettable}

%%%%%%%%%%%%%%%%%%%%%%%%%%%%%%%%%%%%%%%%%%%%%%%%%%%%%%%%%%%%%%%%%%

% Please do not modify or delete this line.
\end{document}

% PROPOSAL SUBMISSION (if not submitting from the online proposal form):
%
%
% Upload this completed proposal latex file and figures at: 
%            https://www.noao.edu/noaoprop/submit/
%
%
% Thank you for your interest in NOAO.  Contact us at
% noaoprop-help@noao.edu if you have any suggestions or comments.
